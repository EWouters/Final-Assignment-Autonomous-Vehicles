\documentclass[a4paper]{article}

\usepackage[a4paper, margin=3cm]{geometry}
\usepackage[english]{babel}
\usepackage[utf8x]{inputenc}
\usepackage{lmodern,textcomp}
\usepackage{amsmath}
\usepackage{graphicx}
\usepackage{caption}
\usepackage{subcaption}
\usepackage{hyperref}
\usepackage{censor}
\usepackage[colorinlistoftodos]{todonotes}

\title{Final Assignment: Autonomous Vehicles}

\author{\censor{Erik Wouters}, Group \censor{13}}

\date{\today}

\begin{document}
\maketitle

\begin{abstract}
\url{https://erikwouters.stackstorage.com/s/byOpy0NoXSEp1xk}
\begin{enumerate}
    \item Please shortly describe the technology. (200 to 400 words)
    \item Take a wider perspective for the chosen technology and preform PESTLE analysis. (200 to 400 words)
    \item Make a comparison between the chosen technology and another technology relevant to the chosen one (the tech delta). Please elaborate why this technology is superior then the comparison technology. (200 to 400 words)
    \item What are (would be) barriers for entry for such technology? (200 to 400 words)
    \item Describe a future scenario where the technology you present is in full use (or what we refer to as a possible second phase in the sequential back casting video). Please describe in detail the new resulting industry structure in this scenario. (200 to 400 words)
    \item What are the new possible business roles that might develop due to the development of the technology? (200 to 400 words)
    \item Who are (or would be) early adopters for this technology? (200 to 400 words)
    \item What is (would be) the strategy for crossing the chasm? (200 to 400 words)
\end{enumerate}
\end{abstract}

\section{Introduction}
\label{sec:introduction}

\section{Wider Perspective}
\label{sec:selecting}

PESTLE

\section{Comparing measurable parameters}
\label{sec:comparing}
%\textit{Make a comparison of the moped relative to the bicycle in the parameters you stated in terms of much less (--), less (-), equal (+-), more (+), much more (++). Elaborate if your assessments are not obvious.}

In the following list the parameters are scored on a Likert scale \cite{likert}. ($++$) indicates that the moped performs much better in this category, whereas ($--$) indicates a much worse performance than the bicycle.

\begin{enumerate}
\item Speed [km/h]
    \begin{enumerate}
    \item[($+$)] The moped is on average faster than the bicycle. A maximum speed for the moped is legally set on $45$ km/h, a fast cyclist can average $25$ km/h.
    \end{enumerate}
\item Initial cost [€]
    \begin{enumerate}
    \item[($--$)] The purchase of a standard moped is more costly than purchasing an average commuter bicycle. However, there are very costly bicycles.
    \end{enumerate}
\item Amount of human effort needed [W/(km/h)]
    \begin{enumerate}
    \item[($++$)] The moped needs less human power at the same speed (which is good).
    \end{enumerate}
\item Weight of the vehicle [kg]
    \begin{enumerate}
    \item[($--$)] The moped weighs more than the bicycle.
    \end{enumerate}
\item Safety [$\star$]
    \begin{enumerate}
    \item[($-$)] The moped is less safe than the bicycle due to higher speeds.
    \end{enumerate}
\item Operating cost [€]
    \begin{enumerate}
    \item[($--$)] The moped has higher operating costs than the bicycle, consisting of maintenance and fuel.
    \end{enumerate}
\item Noise production [dB]
    \begin{enumerate}
    \item[($--$)] The moped produces a rattling combustion engine noise, which leads to noise pollution.
    \end{enumerate}
\end{enumerate}

\section{Technology Barriers}
\label{sec:technology-barriers}

\section{Future Scenario}
\label{sec:future-scenario}

\section{New Business Roles}
\label{sec:new-business-roles}

\section{Early Adopters}
\label{sec:early-adopters}

\section{Crossing the Chasm}
\label{sec:crossing-the-chasm}


\begin{thebibliography}{9}
\bibitem{likert}
  Brinkman, W.-P.(2009). \emph{Design of a Questionnaire Instrument}, Handbook of Mobile Technology Research Methods, ISBN 978-1-60692-767-0, pp. 31-57, Nova Publisher

\end{thebibliography}
\end{document}
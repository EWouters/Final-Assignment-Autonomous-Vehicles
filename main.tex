\documentclass[a4paper]{article}

\usepackage[a4paper, margin=3cm]{geometry}
\usepackage[english]{babel}
\usepackage[utf8x]{inputenc}
\usepackage{lmodern,textcomp}
\usepackage{amsmath}
\usepackage{graphicx}
\usepackage{caption}
\usepackage{subcaption}
\usepackage{hyperref}
\usepackage{censor}
\usepackage[colorinlistoftodos]{todonotes}


\title{Final Assignment: autonomous cars}

\author{\censor{Erik Wouters}, Group \censor{13}}

\date{\today}

\begin{document}
\maketitle

\begin{abstract}
\url{https://erikwouters.stackstorage.com/s/byOpy0NoXSEp1xk}
\begin{enumerate}
    \item Please shortly describe the technology. (200 to 400 words)
    \item Take a wider perspective for the chosen technology and preform PESTLE analysis. (200 to 400 words)
    \item Make a comparison between the chosen technology and another technology relevant to the chosen one (the tech delta). Please elaborate why this technology is superior then the comparison technology. (200 to 400 words)
    \item What are (would be) barriers for entry for such technology? (200 to 400 words)
    \item Describe a future scenario where the technology you present is in full use (or what we refer to as a possible second phase in the sequential back casting video). Please describe in detail the new resulting industry structure in this scenario. (200 to 400 words)
    \item What are the new possible business roles that might develop due to the development of the technology? (200 to 400 words)
    \item Who are (or would be) early adopters for this technology? (200 to 400 words)
    \item What is (would be) the strategy for crossing the chasm? (200 to 400 words)
\end{enumerate}
\end{abstract}

\section{Introduction}
\label{sec:introduction}

The automotive industry started in the 1890s when the technology was ready to combine steam engine and road wagons. In the early 1900s the car sales witnessed the first boom. It had great advantages over the prevailing personal mode of transport at the time, the horse. Today's cars are more safe, durable, convenient and less costly compared to the past.

The automotive industry is currently entering into a new era of technical innovation with the introduction of autonomous cars. This transition comes with new concerns about for instance the possible decline in car ownership, governments imposing new regulations about safety systems and powertrains and a shift in the demographics of car users, and will likely bring back some of the lost features of the horse used to offer.

\section{Wider Perspective}
\label{sec:selecting}

The automotive industry, like any other industry is affected by macro environmental factors like Political, Economical, Social, Technological, Legal, and Environmental factors. The Analysis of these factors can help businesses in developing appropriate strategies in order to minimize the effects of these factors on business \cite{CAMPBELL2005123}.

\subsection{Political}
\label{sec:political}
The automotive industry is, due to its importance to the society and economy, pressured by governments and directly dependent on national and international policies regarding innovation and testing. The testing of new technologies regarding transportation on public roads requires political support to provide an adequate testing climate while minimizing the disturbance imposed on day-to-day operations and still enhancing public safety. Therefore, on all levels of government, testing policies need to be accommodated before the technology can be sufficiently tested in the public space. Policy shaping requires a long process with a lot of actors involved and it usually takes a few years. But with innovative ideas of great impact on peoples safety in the public space, this can even take longer and will require smaller progress steps and more monitoring and fine tuning along the way. This leads to a complex and time-consuming testing process for autonomous vehicle testing, with a slow start for the first-movers and a faster potentially less complex process when a complete set of policies have been approved.

\subsection{Economic}
\label{sec:economic}
Economic prosperity and recessions have a direct influence on the automotive industry and therefore also on the research and development budget for autonomous cars. It is likely that in developed countries the number of cars will stagnate or shrink due to new ownership models like car sharing and mobility as a service \cite{Sivak2008}. The global car sales are predicted to stagnate even in emerging markets in developing countries \cite{Sivak2008}.

\subsection{Social}
\label{sec:social}
The society has a great influence on the progress of autonomous cars by the perceived safety, need, and impact of these vehicles. The market growth is determined by the perception of societal groups and businesses, investors and subsidizers or shareholders on the added value of the new technology. A success, but also failure, can be shared with the world in a split second. Therefore, access to the public is great, but comes also with a need for precautions and need for assurance of delivering all round high quality autonomous cars. 
An important social game changer is the possibility of new car ownership models.

\subsection{Technological}
\label{sec:technological}
Technological factors influencing the uptake of autonomous cars are the speed of sensors and processors, the intelligence of the car, the level of autonomy \cite{KAUR201887}, and the usability of the car/mobility service. Its success is also dependent on the saturation level of autonomous cars in the fleet of all road transport.

\subsection{Legal}
\label{sec:not-gal}
Since the technology is so disruptive, the legal side of who is accountable in case of an accident is still to be determined. The current status is that someone eligible to drive needs to be sitting behind the wheel and needs to put their hands on the wheel at all times and is responsible for the vehicles course of actions. However, with full automation this accountability might partly shift to the manufacturer. At least at the beginning of the introduction of autonomous cars, there will be complex play of legal rights between driver, car owner, insurance companies and manufacturers and others involved about the accountability in certain situations.

\subsection{Environmental}
\label{sec:environmental}

The 7th Environment Action Programme includes the objective to reduce the environmental impact of mobility. Improvements of vehicle efficiency have sustained a stop on the trend of increasing greenhouse gas emissions of transportation, nevertheless a decline is yet to be established \cite{EEA}. Potentially autonomous cars, when electric, can help to shift this trend to a declining one. This is because of platooning, mobility becoming a service and SMART infrastructures and services (aiming to reduce travel times and optimize the vehicle fleet).

\section{Comparing measurable parameters}
\label{sec:comparing}

For the comparison, a set of 7 measurable parameters are chosen. In the following list the parameters are scored on a Likert scale \cite{likert}. ($++$) indicates that the autonomous technology performs much better in this category, whereas ($--$) indicates a much worse performance than the existing non-autonomous cars.

\begin{enumerate}
\item Speed [km/h]
    \begin{enumerate}
    \item[($0$)] the speed of the vehicle is not changed through this new technology.
    \end{enumerate}
\item Initial cost [€]
    \begin{enumerate}
    \item[($--$)] The purchase of a standard car is much less costly than purchasing an autonomous car, especially today.  However, when the technology will become more mature and accepted, the cost difference between the original and autonomous vehicle will decrease.
    \end{enumerate}
\item Amount of human effort needed [W/(km/h)]
    \begin{enumerate}
    \item[($++$)] The autonomous vehicle is aimed to have minimal to zero human effort needed for driving. Compared to the original cars, this is a huge difference.
    \end{enumerate}
\item Weight of the vehicle [kg]
    \begin{enumerate}
    \item[($0$)] The weight of the vehicle is unchanged with respect to the new technology.
    \end{enumerate}
\item Safety [$\star$]
    \begin{enumerate}
    \item[($+$)] The autonomous car, although in its development, is aimed to become safer than the standard vehicle due to the minimization of human involvement at the wheel. However, this is a topic of debate as currently the technology is still developing and not proven safer than standard cars in all traffic environments like not in high density active traffic and low speed environments. In some situations the autonomous vehicle is safer because of its ability to calculate speed and trajectory of other vehicles, pedestrians and cyclists at an intersection or high speed road and as such avoiding collision.
    \end{enumerate}
\item Operating cost [€]
    \begin{enumerate}
    \item[($+$)] The operating costs of an autonomous vehicle will be more economical because of the optimization of energy needed for a certain trajectory. The standard vehicle is in such a case dependent on the human being who might not have all information available to optimize the energy needed. However, also in standard automated vehicles an economical function is implemented which enables drivers to drive more economically. The required maintenance of the two technologies will remain similar. Although less maintenance costs and less costs due to damage or other expenses are implied  when taking into account the higher potential of autonomous cars with regards to optimization of resources and its intended avoidance of collision.
    \end{enumerate}
\item Noise production [dB]
    \begin{enumerate}
    \item[($0$)] Unless the autonomous vehicle will drive electrically or else, the vehicle noise emission will be similar to the standard vehicle.
    \end{enumerate}
\end{enumerate}

\section{Technology Barriers}
\label{sec:technology-barriers}

The barriers for entry for autonomous cars are:

AI needs to be sufficiently advanced to drive the cars.
It needs to be possible for an AI to drive a rally car or an F1 car faster than a human driver, to gain public attention and support.
Autonomous cars need to be much safer than human drivers.
Partial autonomous systems need to be widely accepted and used.
Autonomous cars need to be produced by many manufacturers.
Legislation needs to be in place about autonomous cars.
Insurance needs to accept liability for autonomous car accidents.

\begin{enumerate}
    \item Proven Safety
    \begin{itemize}
        \item The safety of autonomous cars is still dependent on the traffic environment. The more degrees of freedom of other road users, the less autonomous cars perform. 
        \item With this technology developing rapidly, the entry barrier is to when the industry and authorities and potential buyers tend to feel this system is fully safe to use.
    \end{itemize}
    \item Legalization
    \begin{itemize}
        \item Road authorities and legal authorities need to put a legal framework in place in case of accidents or damage as a result of autonomous car crashes. 
        \item Car insurance companies and private car owners will need to be fully aware of their respective responsibilities, liability and rights in case of accidents with autonomous cars.
        \item Because the technology is still in development, these legal frameworks are yet to be established. Also, they currently do not allow for the usage of autonomous cars to their full ability.
        \item Therefore the current legal framework forms an entry barrier to the autonomous vehicle adoption.
    \end{itemize}
    \item Trust
 \end{enumerate}



\section{Future Scenario}
\label{sec:future-scenario}

Everybody drives autonomous cars because it is the cheapest and most convenient way to get around.
We can be productive in the car or enjoy the in car entertainment.
The car hauling company owns the car.
Proffesional users of autonomous cars own their own fleet.
The utilization of cars sky-rockets to over $60\%$, maybe limited by the time spent charging and waiting at night.
Non-autonomous cars will be fewer and fewer. Cities may ban them, highways may force them to use a specific lane.
Less than a third of the amount of cars of today will be on the road.
Many of the existing car manufacturers will have gone bankrupt.
Some of the today existing car manufacturers will have profited from this change by embracing it early enough and will be thriving.

Also:
motorcycles
intercity trains (more)
city planning
office hours

\section{New Business Roles}
\label{sec:new-business-roles}

Manufacturers of electric cars will support this development to gain an additional edge on traditional car makers.
Manufacturers of GPS systems will partner
Manufacturers of hardware that can run AI algorithms will support the transition

\section{Early Adopters}
\label{sec:early-adopters}

The first people to buy self driving cars will be people who drive many kilometers per year, especially on the highways. Their will gain more free time.
There will be new companies in the taxi industry that push the boundaries of autonomous cars.
Traditional car manufacturers who fail to get on board with this trend will try to stop this development to keep their market shares.
Tesla, Volvo, BMW will produce the cars, but companies like comma.ai will sell retro fit kits for existing car models.
Tesla will be the new industry leader.
Existing car manufacturers try to buy startups that have the technology for autonomous cars, or act against the technology.
Electric vehicles will gain a very large market share.

\section{Crossing the Chasm}
\label{sec:crossing-the-chasm}

\bibliography{bibliography}
\bibliographystyle{unsrt}

\end{document}